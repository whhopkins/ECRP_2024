\documentclass[letter, USenglish, 11pt, subfigure]{article}
\usepackage[margin=1in]{geometry}
\newcommand*{\ATLASLATEXPATH}{./}
\usepackage{\ATLASLATEXPATH atlaspackage}
\usepackage{\ATLASLATEXPATH atlasbiblatex}
\usepackage{\ATLASLATEXPATH atlasphysics}
\usepackage{\ATLASLATEXPATH atlasjetetmiss}
\usepackage{\ATLASLATEXPATH ANA-SUSY-2018-12-PAPER-defs}
\usepackage{enumerate}
\newcommand{\mm}{\ensuremath{\mu^{+}\mu^{-}}}
\newcommand{\bsmm}{\ensuremath{\Bs\to\mm}}
\newcommand{\bdmm}{\ensuremath{\Bd\to\mm}}
\usepackage{lineno}
%\linenumbers
\usepackage{wrapfig}
\usepackage{placeins}

\pagestyle{headings}

\title{Early Career Proposal: \\Enabling Calibrations of Machine Learning Approaches (ECMLA)}
\author{Walter Hopkins, Argonne National Laboratory}
\date{}
\begin{document}
\pagenumbering{gobble}

\maketitle
\abstract{
  Results from the Large Hadron Collider (LHC) experiments have verified many of the predictions of the highly successful Standard Model (SM). However, the SM  lacks an explanation for several observed phenomena (e.g., dark matter), motivating the search for Beyond the Standard Model (BSM) physics. The current BSM search strategy uses simplified BSM models (i.e., models with 2-5 parameters) to design BSM search regions. This simplified approach was driven by enhanced BSM physics sensitivity resulting from the large increases of center-of-mass energies during early LHC upgrades. Future LHC upgrades, culminating in the high-luminosity (HL) LHC, will no longer include significant increases in energy, but will result in a tenfold increase of the current data set.
The HL-LHC comes with challenges; in particular, two challenges faced by the ATLAS experiment are (1) the need to exhaustively probe the large high-dimensional data set for evidence of BSM physics and (2) the computational requirements for producing large amounts of simulated data to estimate SM backgrounds. Machine learning (ML) and the upcoming exascale High Performance Computing (HPC) resources will provide promising tools to tackle these HL-LHC computing challenges. The proposed research focuses on developing an automated BSM  search strategy, using complex theory models (i.e. with more than ten parameters), to probe the far reaches of the HL-LHC data set. Furthermore, to estimate the SM background in these regions, the research will aim to improve the computational speed of ATLAS detector simulations with ML. The search strategy will be generated by ML clustering algorithms that will group theory models with similar experimental signatures together. The computational speed of ATLAS simulations will be improved by tuning simulation configurations while ML algorithms will ensure minimal loss of accuracy. By systematically constructing search regions and accelerating simulations, this approach will enable ATLAS to maximally exploit the HL-LHC data set for BSM physics discovery.
}

\end{document}