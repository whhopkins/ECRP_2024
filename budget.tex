\documentclass[letter, USenglish, 11pt, subfigure]{article}
\usepackage[margin=1in]{geometry}
\usepackage{pdfpages}
\newcommand*{\ATLASLATEXPATH}{./}
\usepackage{\ATLASLATEXPATH atlaspackage}
\usepackage{\ATLASLATEXPATH atlasbiblatex}
\usepackage{\ATLASLATEXPATH atlasphysics}
\newcommand{\mm}{\ensuremath{\mu^{+}\mu^{-}}}
\newcommand{\bsmm}{\ensuremath{\Bs\to\mm}}
\newcommand{\bdmm}{\ensuremath{\Bd\to\mm}}
\usepackage{lastpage}
\usepackage{fancyhdr}
\pagestyle{fancy}
\lhead{Budget justification}
\rhead{Walter Hopkins}
\fancyfoot[C]{\thepage\ of \pageref*{LastPage}}

\fancypagestyle{firststyle}
{
   \fancyhf{}
   \fancyfoot[C]{\thepage\ of \pageref*{LastPage}}
   \renewcommand{\headrulewidth}{0pt} % removes horizontal header line
 }
\renewcommand{\thesection}{\Alph{section}}

\addbibresource{proposal_WHopkins.bib}
%\pagestyle{headings}
\title{Argonne National Laboratory\\Budget Justification: DE-FOA-0003176}
\author{Enabling Calibrations of Machine Learning Approaches (ECMLA)\\ Walter Hopkins, Physicist\\
}
\date{}

\begin{document}

\maketitle
\thispagestyle{firststyle}

\section{Senior/Key Person}
\label{subsec:keyPerson}
The PI is the only senior person that will be funded and will be funded at an average of 0.5 FTE. The average cost per budget period is \$138,883 of which \$107,661 is the requested salary and \$31,222 comes from fringe benefits. The total direct costs associated with the PI's effort is \$694,414.

\section{Other Personnel}
\label{subsec:personnel}
Other personnel will include an average of 2.0 FTE of postdocs. One postdocs, at 1.0 FTE, will be hired at the beginning the first budget period while another postdoc, also at 1.0 FTE, will be hired six months later. The contract of the postdoc will end at the end of the fourth and beginning of the last budget period. Funding for graduate students is not included in the proposed budget because laboratories typically do not fund graduate students. However, the PI will collaborate with local universities through the ATLAS Center and work with students through DOE Office of Science Graduate Student Research and the Science Undergraduate Laboratory Internship programs.

\section{Equipment}
\label{subsec:Equipment}
The requested equipment consists of two laptops for the two postdocs that will be funded for the proposed work and is expected to have a total direct cost of \$6,418. 

\section{Travel}
\label{subsec:travel}
Travel will mainly consist of trips to CERN for the PI and postdocs. Based on previous experience, travel to CERN costs \$3000 per trip. About two trips per year and per person will be made to CERN. These trips will be to attend the ATLAS Physics, Machine Learning Forum, and E/gamma meetings which are held multiple times a year. The Physics meetings are essential in communicating with ATLAS collaborators and making sure the proposed methods have an impact on ATLAS.

In addition to trips to CERN, two domestic trips to conferences to highlight results are foreseen every year. Based on previous experience, these trips cost \$1500. Table~\ref{tab:yearlyTravelBudget} shows the breakdown of travel costs for each budget period. Examples of conferences where new physics search results can be shown are LHCP or ICHEP. These are international conferences that can occur anywhere in the world but the ones that will be domestic will be chosen. For the machine learning work, conferences such as CHEP and ACAT will be considered. 

\begin{table}[!htpb]
  \begin{center}  
    \caption{Travel budget and trips per period. Indirect costs are not included.}
    \label{tab:yearlyTravelBudget}
  \begin{tabular}{llll}
  \hline
  {} & Domestic trips & International trips & Travel Funds \\
  \hline
  Budget period 1 &              2 &                   5 &       \$18,000 \\
  Budget period 2 &              2 &                   5 &      \$18,000 \\
  Budget period 3 &              2 &                   5 &      \$18,000 \\
  Budget period 4 &              2 &                   5 &      \$18,000 \\
  Budget period 5 &              3 &                   5 &      \$19,500 \\
  \hline
  Total &              11 &                   25 &      \$91,500 \\
  \hline
\end{tabular}

  \end{center}
\end{table}

\section{Participant/Trainee Support Costs}
\label{subsec:trainee}
Not applicable.

\section{Other Direct Costs}
\label{subsec:otherDirects}
There are no other direct costs other than the personnel, equipment, and travel costs listed above.

\section{Direct Costs}
\label{subsec:directs}
The total direct costs are \$1,802,920 of which \$1,705,002 is in salaries for the PI and postdocs, \$6,418 is for equipment, and \$91,500 is for travel.

\section{Other Indirect Costs}
\label{subsec:otherIndirects}
Argonne uses several indirect rates and basis for distribution of indirect expenses. The different indirect rates for FY2024 and beyond are shown in Table~\ref{tab:indirectRates}. The total indirect costs, which depend on the mix of costs from various expense categories, are \$947,080 which is equivalent to a combined total rate of 53\%. In addition to these indirect rates that are applied to direct costs, an LDRD indirect cost is added but is applied to the total cost (direct+non-LDRD indirect costs).

\begin{table}[!htpb]
  \begin{center}  
    \caption{Indirect cost rates for different expense categories.}
    \label{tab:indirectRates}
    \begin{tabular}{lll}
      \hline
      {} & FY2023 rate & Applied to  \\
      \hline
      Common Support & 33.4\% & Effort and other non-M\&S costs\\
      IGPP/IGPE & 1.1\% &  Effort and other non-M\&S costs \\
      M\&S & 6.4\% & Materials and services\\
      General and Admin & 7.2\% & All direct costs\\
      Physical Sciences and Engineering ALD & 4.6\% & All direct costs\\
      LDRD & 3.7\% & All costs (direct and non-LDRD indirect)\\
      \hline
    \end{tabular}
  \end{center}
\end{table}


%\clearpage
\section{Total Direct and Indirect Costs}
\label{subsec:totalCosts}
The total cost of this proposal is \$2,750,000. 


\clearpage

%\includepdf[scale=0.8,pagecommand=\section{Contracting Officer’s concurrence to participate}]{CO_approval_printed}

\end{document}

