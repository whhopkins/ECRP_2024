\documentclass[letter, USenglish, 11pt, subfigure]{article}
\usepackage[margin=1in]{geometry}
\usepackage{pdfpages}
\newcommand*{\ATLASLATEXPATH}{./}
\usepackage{\ATLASLATEXPATH atlaspackage}
\usepackage{\ATLASLATEXPATH atlasbiblatex}
\usepackage{\ATLASLATEXPATH atlasphysics}
\newcommand{\mm}{\ensuremath{\mu^{+}\mu^{-}}}
\newcommand{\bsmm}{\ensuremath{\Bs\to\mm}}
\newcommand{\bdmm}{\ensuremath{\Bd\to\mm}}
\usepackage{lastpage}
\usepackage{fancyhdr}
\pagestyle{fancy}
\lhead{Budget justification}
\rhead{Walter Hopkins}
\fancyfoot[C]{\thepage\ of \pageref*{LastPage}}

\fancypagestyle{firststyle}
{
   \fancyhf{}
   \fancyfoot[C]{\thepage\ of \pageref*{LastPage}}
   \renewcommand{\headrulewidth}{0pt} % removes horizontal header line
 }
\renewcommand{\thesection}{\Alph{section}}

\addbibresource{proposal_WHopkins.bib}
%\pagestyle{headings}
% ebud: 39999.1
\title{Budget justification}
\author{Enabling Calibrations of Machine Learning Approaches (ECMLA)\\ Walter Hopkins, Assistant Physicist\\
}
\date{}

\begin{document}

\maketitle
\thispagestyle{firststyle}

\section{Senior/Key Person}
\label{subsec:keyPerson}
The PI is the only senior person that will be funded. The PI will be funded at an average of 0.50 FTE. Table~\ref{tab:yearlyBudget} shows the breakdown of the PIs FTE for each budget period.

\begin{table}[!htpb]
  \begin{center}
    \caption{Effort costs for each period. Indirect costs are not included.}
      \label{tab:yearlyBudget}
  %\scriptsize
  \begin{tabular}{lrrrr|rrrr}
\hline
 & \multicolumn{4}{c|}{Senior (PI)} & \multicolumn{4}{c}{Postdoctoral} \\
{} & Months & Salary & Fringe &  Total & Months & Salary & Fringe & Total \\
\hline
Period 1 &       9.0 &  \$136,246 &   \$40,874 &  \$177,120 &           12.5 &        \$77,778 &        \$19,055 &       \$96,833 \\
Period 2 &       7.0 &  \$122,733 &   \$36,820 &  \$159,553 &           18.0 &       \$116,541 &        \$28,553 &      \$145,094 \\
Period 3 &       6.0 &  \$108,992 &   \$32,698 &  \$141,690 &           29.5 &       \$198,431 &        \$48,615 &      \$247,046 \\
Period 4 &       6.0 &  \$112,924 &   \$33,877 &  \$146,801 &           17.0 &       \$118,806 &        \$29,108 &      \$147,914 \\
Period 5 &       6.0 &  \$117,002 &   \$35,100 &  \$152,102 &           12.0 &        \$87,132 &        \$21,347 &      \$108,479 \\
\hline
Total    &      34.0 &  \$597,897 &  \$179,369 &  \$777,266 &           89.0 &       \$598,688 &       \$146,678 &      \$745,366 \\
\hline
\end{tabular}

  \end{center}
\end{table}

\section{Other Personnel}
\label{subsec:personnel}
Other personnel will include an average of 2.0 FTE of postdocs. Two postdocs, each at 1.0 FTE, will be hired at the beginning the first budget period. The contract of the postdoc will end at the end of the fourth and beginning of the last budget period. Table~\ref{tab:yearlyBudget} shows the breakdown of the postdoc FTEs for each budget period. Funding for graduate students is not included in the proposed budget because laboratories typically do not fund graduate students. However, the PI will collaborate with local universities through the ATLAS Center and work with students through DOE Office of Science Graduate Student Research and the Science Undergraduate Laboratory Internship programs.

\clearpage
\section{Equipment}
\label{subsec:Equipment}
The requested equipment consists of two laptops for the two postdocs that will be funded for the proposed work and is expected to have a total cost of \$7,360. 

\section{Travel}
\label{subsec:travel}
Travel will mainly consist of trips to CERN for the PI and postdocs. Based on previous experience, travel to CERN costs \$3000 per trip. About two trips per year and per person will be made to CERN. These trips will be to attend the ATLAS Physics, Machine Learning Forum, and E/gamma meetings which are held multiple times a year. The Physics meetings are essential in communicating with ATLAS collaborators.

In addition to trips to CERN, three domestic trips to conferences to highlight results are foreseen every two years. Based on previous experience, these trips cost \$1500. Table~\ref{tab:yearlyTravelBudget} shows the breakdown of travel costs for each budget period. {\color{red} NEEDS UPDATING Examples conferences where new physics search results can be shown are SUSY or ICHEP. These are international conferences that can occur anywhere in the world but the ones that will be domestic will be chosen. For the detector simulation work, conferences such as CHEP and ACAT will be considered. These trips are planned for budget period two, three and four. Budget period two will include one trip for the PI and one for the 0.5 FTE ATLAS postdoc to highlight the early detector simulation results based on simplified geometries. Budget period three will include one trip for the PI and two trips for the two postdoc that were hired to focus on the proposed work. These trips will be to show the results of the pMSSM interpretation and detector simulations with more complex geometries. Finally, budget period five will include one trip for the PI, one for the 1.0 FTE postdoc, and one for the 0.5 FTE ATLAS postdoc. These trips are meant to share the results of the completed new physics search and early ATLAS simulation results. }

\begin{table}[!htpb]
  \begin{center}  
    \caption{Travel budget and trips per period. Indirect costs are not included.}
    \label{tab:yearlyTravelBudget}
  \begin{tabular}{llll}
  \hline
  {} & Domestic trips & International trips & Travel Funds \\
  \hline
  Budget period 1 &              2 &                   5 &       \$18,000 \\
  Budget period 2 &              2 &                   5 &      \$18,000 \\
  Budget period 3 &              2 &                   5 &      \$18,000 \\
  Budget period 4 &              2 &                   5 &      \$18,000 \\
  Budget period 5 &              3 &                   5 &      \$19,500 \\
  \hline
  Total &              11 &                   25 &      \$91,500 \\
  \hline
\end{tabular}

  \end{center}
\end{table}

\section{Participant/Trainee Support Costs}
\label{subsec:trainee}
Not applicable.

\section{Other Direct Costs}
\label{subsec:otherDirects}
There are no other direct costs other than the personnel, equipment, and travel costs listed above.

\section{Direct Costs}
\label{subsec:directs}
The total direct costs are \$1,615,156 of which \$1,522,632 is in salaries for the PI and postdocs, \$8,524 is for equipment, and \$84,000 is for travel.

\section{Other Indirect Costs}
\label{subsec:otherIndirects}
Argonne uses several indirect rates and basis for distribution of indirect expenses. The different indirect rates for FY2024 and beyond are shown in Table~\ref{tab:indirectRates}. The total indirect costs, which depend on the mix of costs from various expense categories, per budget period are shown in Table~\ref{tab:total} and come to a combined total rate of { \color{red} 50\% . Not considering LDRD, the total indirect rate for effort is 50.9\% while the rate for equipment and travel is 20.8\% for all budget periods NEEDS TO UPDATED}. In addition to these indirect rates that are applied to direct costs, an LDRD indirect cost is added but is applied to the total cost (direct+non-LDRD indirect costs).

\begin{table}[!htpb]
  \begin{center}  
    \caption{Indirect cost rates for different expense categories.}
    \label{tab:indirectRates}
    \begin{tabular}{lll}
      \hline
      {} & FY2023 rate & Applied to  \\
      \hline
      Common Support & 33.4\% & Effort and other non-M\&S costs\\
      IGPP/IGPE & 1.1\% &  Effort and other non-M\&S costs \\
      M\&S & 6.4\% & Materials and services\\
      General and Admin & 7.2\% & All direct costs\\
      Physical Sciences and Engineering ALD & 4.6\% & All direct costs\\
      LDRD & 3.7\% & All costs (direct and non-LDRD indirect)\\
      \hline
    \end{tabular}
  \end{center}
\end{table}

% \begin{table}[!htpb]
%   \begin{center}  
%     \caption{Indirect costs per budget period.}
%       \label{tab:indirects}
%   \begin{tabular}{lrr}
\hline
{} & Indirect costs    \\
  \hline
  Budget period 1 &         \$139,873      \\
  Budget period 2 &       \$152,000     \\
  Budget period 3 &         \$206,118      \\
  Budget period 4 &         \$186,990     \\
  Budget period 5 &        \$151,231     \\\hline
  Total           &             \$836,212  \\
  \hline
\end{tabular}

%   \end{center}
% \end{table}
%\clearpage
\section{Total Direct and Indirect Costs}
\label{subsec:totalCosts}
The total cost of this proposal is \$2,749,999. The direct, indirect, and total cost per budget period can be found in Table~\ref{tab:total}.

\begin{table}[!htpb]
  \begin{center}  
    \caption{Direct and indirect costs per budget period.}
      \label{tab:total}
  \begin{tabular}{llll}
\hline
{} &   Direct costs & Indirect costs &     Total cost \\
\hline
Budget period 1 &    \$283,899 &     \$139,873 &    \$423,772 \\
Budget period 2 &    \$301,785 &     \$152,000 &    \$453,785 \\
Budget period 3 &    \$408,103 &     \$206,118 &    \$614,221 \\
Budget period 4 &    \$370,579 &     \$186,990 &    \$557,569 \\
Budget period 5 &    \$299,422 &     \$151,231 &    \$450,653 \\\hline
Total           &  \$1,663,788 &           \$836,212 &  \$2,500,000 \\
\hline
\end{tabular}

  \end{center}
\end{table}
\clearpage

%\includepdf[scale=0.8,pagecommand=\section{Contracting Officer’s concurrence to participate}]{CO_approval_printed}

\end{document}

